% =============================================================================

We evaluated each ISE variant by implementing the AES-128
$\ALG{Enc}$,
$\ALG{Dec}$
{\em plus}
$\ALG{Enc-KeyExp}$
and
$\ALG{Dec-KeyExp}$.
We used a {\em non}-ISE 
T-table based
implementation as a baseline.
The variants which assume  $\RVXLEN = 32$
(\ISE{1}, \ISE{2},     \ISE{3}, and \ISE{5})
used a    rolled strategy wrt. loops:
 \ISE{1}, \ISE{2},              and \ISE{5}
used  $1$ round  per iteration,
whereas
                       \ISE{3}
used $2$ rounds per iteration
to avoid needless register move operations.
The variant  which assumes $\RVXLEN = 64$
(\ISE{4})
used an unrolled strategy.
In all cases the state is naturally aligned,\footnote{%
RISC-V does not mandate support for misaligned loads and stores, so
aligning the state this way ensures the best performance across all
cores.
} meaning any input (resp. output) can be loaded (resp. stored) 
using 
$4$ \VERB{lw} instructions on a $32$-bit core
or
$2$ \VERB{ld} instructions on a $64$-bit core.

\REFTAB{tab:eval:sw:size} 
records the
memory footprint (i.e., code footprint and static data footprint)
of each software implementation.
Where an entry for
\ALG{Dec-KeyExp}
is     zero, this implies that
$\ALG{Enc-KeyExp} =    \ALG{Dec-KeyExp}$
so there is no overhead.
Where an entry for
\ALG{Dec-KeyExp}
is non-zero, this implies that
$\ALG{Enc-KeyExp} \neq \ALG{Dec-KeyExp}$,
and the equivalent inverse cipher construction~\cite[Section 5.3.5]{FIPS:197}
is used.
This allows $\ALG{Dec-KeyExp}$ to
call $\ALG{Enc-KeyExp}$,
then
perform some additional post processing,
with the quoted footprint therefore reflecting the latter only.  
\REFTAB{tab:eval:sw:perf:2}
and
\REFTAB{tab:eval:sw:perf:1}
record
instruction (i.e., iret) and cycle counts
of each implementation,
as executed on the \CORE{2} and \CORE{1} cores respectively.

% =============================================================================
