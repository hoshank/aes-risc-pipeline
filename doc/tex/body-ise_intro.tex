% =============================================================================

\REFSEC{sec:bg:aes_impl_ise}
outlined a range ISE designs, which constitute a large design space of
options that we {\em could} consider.  To narrow the design space into
those we {\em do} consider, we use the requirements as outlined below:

\begin{requirement}
The ISE should align with the wider RISC-V ethos and design principles.
Foremost, this means it should 
1) favour simple building-block operations,
   and
2) permit an instruction format with at most $2$ source and $1$ destination register address.
\end{requirement}

\begin{requirement}
The ISE should operate on 
the RISC-V general-purpose scalar register file 
(i.e., st. $w \in \SET{32,64}$),
vs. 
any                        vector register file
(e.g., $w \ge 128$).
Note that this requirement excludes most standard ISEs outlined in 
\REFSEC{sec:bg:aes_impl_ise}.
\end{requirement}

\begin{requirement}
The ISE should introduce no
special-purpose       architectural state, 
nor rely on
special-purpose micro-architectural state
(e.g., caches or scratch-pad memory).
\end{requirement}

\begin{requirement}
The ISE should afford data-oblivious execution of AES, and thus prevent 
(digital) side-channel attacks based on execution time 
(e.g., stemming from accesses to the S-box).
\end{requirement}

% TODO: should we factor in memory footprint (and if so, how)

\begin{requirement}
The ISE must be efficient, in terms of improvement in latency per unit
unit of hardware required: this balances the value in both metrics, vs.
exclusively favouring one or the other.
\end{requirement}

\noindent
Overall, the requirements combine to intentionally target the ISE at 
 low(er)-end,
resource-constrained (e.g., embedded) platforms.  
We view such a focus as reasonable, because existing work on adding
cryptographic support to the
standard V 
(or vector) 
extension~\cite[Section 21]{RV:ISA:I:19}
already caters for
high(er)-end
alternatives.

We arrive at five ISE variants using the requirements, the description of 
which is summarised by 

\begin{center}
\begin{tabular}{|c|ccc|}
\hline
Mnemonic & Intuitive                  & Technical               & Diagrammatic               \\
\hline
\ISE{1}  & \REFSEC{sec:ise:design:v1} & \REFAPPX{sec:pseudo:v1} & \REFFIG{fig:ise:design:v1} \\
\ISE{2}  & \REFSEC{sec:ise:design:v2} & \REFAPPX{sec:pseudo:v2} & \REFFIG{fig:ise:design:v2} \\
\ISE{3}  & \REFSEC{sec:ise:design:v3} & \REFAPPX{sec:pseudo:v3} & \REFFIG{fig:ise:design:v3} \\
\ISE{4}  & \REFSEC{sec:ise:design:v4} & \REFAPPX{sec:pseudo:v4} &                            \\
\ISE{5}  & \REFSEC{sec:ise:design:v5} & \REFAPPX{sec:pseudo:v5} & \REFFIG{fig:ise:design:v5} \\
\hline
\end{tabular}
\end{center}

\noindent
That is, for each variant we include
1) an 
   intuitive 
   description in one of the following \SEC[s],
2) a
   technical
   description
   (i.e., a complete list of instructions and their semantics, plus examples of their use when implementing AES)
   in a dedicate \APPX,
   and
3) a 
   diagramatic
   description
   in \REFFIG{fig:ise:design}.

% =============================================================================
