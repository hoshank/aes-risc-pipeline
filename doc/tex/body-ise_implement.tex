% =============================================================================

The evaluation of each ISE considers two different RISC-V compliant base
micro-architectures, which constitute two different host cores:

\begin{itemize}
\item The \CORE{2}\footnote{%
        \ifbool{anonymous}{Details of this core have been anonymised to comply with the TCHES submission guidelines.}{\url{https://github.com/scarv/scarv}}
      } core 
      supports the 
      RV32IMC 
      instruction set, i.e.,
      the 
             $32$-bit~\cite[Section 2]{RV:ISA:I:19} 
      base integer ISA plus 
      standard 
      Multiplication ~\cite[Section  7]{RV:ISA:I:19}
      and
      Compressed ~\cite[Section 16]{RV:ISA:I:19}
      extensions.
      Per the block diagram shown in~\REFFIG{fig:core:2:normal},
      the core 
      executes instructions using a $5$-stage, in-order pipeline.
      No branch prediction is supported.
      There are two memory interfaces for instruction fetch and data memory
      accesses.
      No instruction or data caches are supported.
      The core implements various performance counters,
      and
      elements of the
      RISC-V Privileged Resource Architecture (PRA)~\cite[Chapter 3]{RV:ISA:II:17}
      related to exception and interrupt handling.

\item The \CORE{1}~\cite{rocket:16} 
        core
      executes instructions using a $5$-stage, in-order pipeline
      which is highly configurable.
      We take advantage of this, considering two variants whose
      exact configuration is outlined in
      \REFFIG{fig:rocket:32} 
      and 
      \REFFIG{fig:rocket:64}:
      the variants represent single $32$-bit and $64$-bit cores respectively,
      and so
      support  the 
      RV32IMC 
      (resp. RV64IMC)
      instruction set, i.e.,
      the 
             $32$-bit~\cite[Section 2]{RV:ISA:I:19} 
      (resp. $64$-bit~\cite[Section 5]{RV:ISA:I:19})
      base integer ISA plus 
      standard 
      Multiplication ~\cite[Section  7]{RV:ISA:I:19}
      and
      Compressed ~\cite[Section 16]{RV:ISA:I:19}
      extensions.
      Each varient is configured to support
      an instruction cache, 
      a  data        cache,
      and
      a  branch prediction mechanism,
      but 
      no floating-point support.

\end{itemize}

\noindent
To support a each ISE, two modifications were made to each host core:
the instruction decoder was modified to support
operand selection
and
an AES Functional Unit (AES-FU) was added to support execution of
ISE instructions.
The \CORE{2} core integrates the AES-FU directly into the pipeline,
while
the \CORE{1} core accesses   the AES-FU via the
Rocket Custom Coprocessor (RoCC)~\cite[Section 4]{rocket:16}
interface.
Due to \REFREQ{req:2}, 
specifically that each instruction uses at most $2$ source and $1$ destination register, 
neither micro-architecture required further structural alteration.
A synthesis-time parameter was used to switch between different 
ISEs.

% =============================================================================
