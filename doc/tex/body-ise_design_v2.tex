% =============================================================================

By reproducing~\cite[Section 4.3]{TilGro:06},
\ISE{2}
assumes 
$w = 32$
and adopts a 
column-packed 
representation of state and round key matrices.
As detailed in
\REFFIG{fig:mnemonics:v2}
and
\REFFIG{fig:pseudo:v2},
it adds
$ 4$
instructions ($2$ for encryption, $2$ for decryption).
For example
\VERB{saes.v2.encs}
applies \AESFUNC{SubBytes}  to elements in a packed column,
and
\VERB{saes.v2.encm}
applies \AESFUNC{MixColumn} to             a packed column (which is optionally rotated).
Both 
\VERB{saes.v2.encs}
and
\VERB{saes.v2.encm}
use an instruction format with $2$ source and $1$ destination register address.
\ISE{2} improves \ISE{1} by applying \AESFUNC{ShiftRows} 
{\em implicitly}:
this is supported by careful indexing of elements in source and destination
columns during application of \AESFUNC{SubBytes} and \AESFUNC{MixColumns},
and also permits
\VERB{saes.v2.encs}
to be used within the key schedule.

As detailed in
\REFFIG{fig:round:v2},
using this variant to implement AES encryption requires
$16$
instructions per round:
$ 4$ \VERB{saes.v1.encs} instructions to apply \AESFUNC{SubBytes},
$ 4$ \VERB{saes.v1.encm} instructions to apply \AESFUNC{MixColumns},
$ 4$ \VERB{ lw}          instructions to load the round key,
and
$ 4$ \VERB{xor}          instructions to apply \AESFUNC{AddRoundKey}.
However, recall that the $Nr$-th round omits \AESFUNC{MixColumns}: in this
case, application of \AESFUNC{ShiftRows} must be completed 
{\em explicitly}
via additional 
$12$
instructions.

% =============================================================================
