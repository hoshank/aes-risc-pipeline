% =============================================================================

Each ISE variant was implmented and evaluated on the host cores
described in \REFSEC{sec:design}.
The 32-bit designs (V1,V2,V3,V5) were implemented on both the
\CORE{1} and \CORE{2} cores.
The 64-bit design (v4) was only evaluated on the 64-bit configuration
of the \CORE{2} core.

Table \ref{tab:eval:hw}
shows the hardware implementation costs, while
Table \ref{tab:eval:sw}
shows performance and code size results for
each ISE, with software only versions of AES used as a baseline.

For variants 1, 2 and 5, two implementations are evaluated.
The ``Size" optimised implementations instantiate only a single
Forward/Inverse SBox and MixColumns circuit and take multiple cycles
to produce a result.
The ``Latency" optimised implementations instantiate $4$ SBox and
MixColumn circuits to produce their results in a single processor cycle.

The ISE Size column of Table \ref{tab:eval:hw} 
records the size in NAND2 equivalent gates of each variant,
instantiated independently from any wider system.
The LTP column gives the Longest Topological Path of the synthesised
functional unit circuit from input to combinatorial output.
The \CORE{2} Size column gives the size in NAND2 equivilent gates of the
\CORE{2}, with the various AES functional units integrated.
The ``Baseline" row gives the size of the core without any of the
ISEs integrated.
We found that none of the proposed ISEs affected the critical
path of the \CORE{2} core.

% ------------------------------------------------------------

\begin{table}
\centering
\begin{tabular}{lrrrr}
Variant     & ISE Size & ISE LTP & \CORE{2} Size & Size Overhead \\ \hline
Baseline    & -        & -       & 37375         & -             \\
V1 (Latency)& 3472     & 19      & 41723         & $  \%$        \\
V2 (Latency)& 3547     & 19      & 41199         & $  \%$        \\
V5 (Latency)& 4121     & 22      & 42070         & $  \%$        \\
V3          & 1157     & 30      & 38610         & $  \%$        \\
V4          &          &         &               & $  \%$        \\
V1 (Size)   & 2174     & 22      & 40161         & $  \%$        \\
V2 (Size)   & 1381     & 21      & 38885         & $  \%$        \\
V5 (Size)   & 1927     & 23      & 39251         & $  \%$        \\
\end{tabular}
\caption{
Hardware implementation costs based on the 32-bit \CORE{2} CPU core.
Gate counts and topological path lengths are obtained using the
Yosys\cite{yosys} tool suite.
}
\label{tab:eval:hw}
\end{table}


To evaluate the software performance, we implemented AES 128, and
measure the static code size, instruction execution counts and cycle
counts of the Key Schedule, Encrypt and Decrypt functions.
We also separate generation of the KeySchedule for Encrypt and Decrypt.

\REFTAB{tab:eval:sw:size} shows the static code size for each
function.
We see that... ({\bf TODO}).

\REFTAB{tab:eval:sw:perf} gives cycle and instruction counts for each
variant.
Each implementation uses word-aligned state, meaning the input blocks
can be loaded with $4$ load-word instructions on $32$-bit host cores,
or $2$ load-double instructions on the $64$-bit host cores.

\begin{table}
\centering
\begin{tabular}{l|cc|cc|cc|cc}
 &
\multicolumn{2}{c}{Key Schedule (Enc)} &
\multicolumn{2}{c}{Key Schedule (Dec)} &
\multicolumn{2}{c}{Encrypt} &
\multicolumn{2}{c}{Decrypt}  \\
Variant & .text & .data & .text & .data & .text & .data & .text & .data  \\ \hline
Byte    &       &       &       &       &       &       &       &        \\
TTable  &       &       &       &       &       &       &       &        \\
V1      &       &       &       &       &       &       &       &        \\
V2      &       &       &       &       &       &       &       &        \\
V3      &       &       &       &       &       &       &       &        \\
V4      &       &       &       &       &       &       &       &        \\
V5      &       &       &       &       &       &       &       &        \\
\end{tabular}
\caption{
Static code size metrics for each variant.
}
\label{tab:eval:sw:size}
\end{table}


%
% Commented out this table because in the real world, you'd make sure that
% your state is well aligned!
%
% \begin{table}[pt]
% \centering
% \begin{tabular}{l|cc|cc|cc|cc}
% & \multicolumn{2}{c}{\begin{tabular}[c]{@{}c@{}}KeyExpand\\ Encrypt\end{tabular}} 
% & \multicolumn{2}{c}{\begin{tabular}[c]{@{}c@{}}KeyExpand\\ Decrypt\end{tabular}}
% & \multicolumn{2}{c}{\begin{tabular}[c]{@{}c@{}}AES 128\\ Encrypt\end{tabular}}
% & \multicolumn{2}{c}{\begin{tabular}[c]{@{}c@{}}AES 128\\ Decrypt\end{tabular}} \\
% Variant     &  iret & cycles & iret & cycles & iret & cycles & iret & cycles \\ \hline
%  Byte       &  926  & 3887   & 926  & 3886   & 4228 & 7061   & 7652 & 11587 \\
%  TTable     &  481  & 591    & 1762 & 2238   & 1013 & 1144   & 1013 & 1118  \\
% V1 (Latency)&  249  & 369    & 255  & 386    & 593  & 698    & 593  & 707   \\
% V2 (Latency)&  249  & 382    & 386  & 694    & 296  & 404    & 297  & 404   \\
% V3          &  269  & 388    & 719  & 1145   & 321  & 413    & 321  & 408   \\
% V5 (Latency)&  383  & 524    & 389  & 541    & 308  & 408    & 308  & 409   \\
% V1 (Size)   &  249  & 409    & 255  & 426    & 593  & 858    & 593  & 875   \\
% V2 (Size)   &  249  & 412    & 386  & 832    & 296  & 641    & 297  & 641   \\
% V5 (Size)   &  383  & 554    & 389  & 571    & 308  & 660    & 308  & 650   \\
% \end{tabular}
% \caption{
% Performance metrics for byte aligned state.
% }
% \label{tab:eval:sw:perf:byte}
% \end{table}

\begin{table}
\centering
\begin{tabular}{l|cc|cc|cc|cc}
& \multicolumn{2}{c}{\begin{tabular}[c]{@{}c@{}}KeyExpand\\ Encrypt\end{tabular}} 
& \multicolumn{2}{c}{\begin{tabular}[c]{@{}c@{}}KeyExpand\\ Decrypt\end{tabular}}
& \multicolumn{2}{c}{\begin{tabular}[c]{@{}c@{}}AES 128\\ Encrypt\end{tabular}}
& \multicolumn{2}{c}{\begin{tabular}[c]{@{}c@{}}AES 128\\ Decrypt\end{tabular}} \\
Variant     &  iret & cycles & iret & cycles & iret & cycles & iret & cycles \\ \hline
 Byte       & 926 & 3887 & 926 & 3886 & 4228 & 7061 & 7652 & 11587\\
 TTable     & 445 & 555 & 1726 & 2202 & 953 & 1078 & 953 & 1058   \\
V1 (Latency)& 213 & 333 & 219 & 350 & 533 & 635 & 533 & 647\\
V2 (Latency)& 213 & 344 & 350 & 656 & 236 & 343 & 237 & 343\\
V5 (Latency)& 347 & 489 & 353 & 506 & 248 & 346 & 248 & 349\\
V1 (Size)   & 213 & 373 & 219 & 390 & 533 & 795 & 533 & 815  \\
V2 (Size)   & 213 & 374 & 350 & 794 & 236 & 580 & 237 & 580  \\
V3          & 233 & 352 & 683 & 1109 & 261 & 351 & 261 & 346 \\
V4          &     &     &     &      &     &     &     &     \\
V5 (Size)   & 347 & 519 & 353 & 536 & 248 & 598 & 248 & 590  
\end{tabular}
\caption{
Performance metrics for word aligned state.
}
\label{tab:eval:sw:perf}
\end{table}

% ------------------------------------------------------------

We can see from the tables that ({\bf TODO}).

% =============================================================================
