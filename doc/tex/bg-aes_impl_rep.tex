% =============================================================================

It should be clear that an element in $\F_{2^8}$ can be represented by an
$8$-bit byte: 
$
x_i \in \SET{ 0, 1 } ,
$
the $i$-th bit of $x$ for $0 \leq i < 8$, naturally represents the $i$-th 
polynomial coefficient.

Beyond this, however, several options exist for representing the state and
round key matrices.  The most direct option would be termed
 array-based (or unpacked):
one simply represents the matrix as a $16$-element array of $8$-bit bytes, 
each of which represents a field element.
Although FIPS-197~\cite{FIPS:197} defines a word to be st. $w = 32$, let
$w$ generally denote the natural word size, measured in bits, of whatever
platform the implementation is based on.
Where $w >     16$,
it is plausible to pack
more than one field element 
into each word.  
Pertinent cases are st. multiple elements from the same row (resp. column) 
of a matrix are packed into each word:
we term these
   row-packed  
and
column-packed
representations respectively.
Where $w \geq 128$, 
it is plausible to pack
an entire matrix
into each word: 
we term this a 
 fully-packed 
representation.

% =============================================================================
