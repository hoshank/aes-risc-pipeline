% =============================================================================

Each ISE variant was implemented and integrated into the $3$ host cores 
described in \REFSEC{sec:ise:imp}.
The variants which assume  $w = 32$
(\ISE{1}, \ISE{2}, \ISE{3}, and \ISE{5}) 
were evaluated wrt.
{\em both}
on the
$32$-bit \CORE{2} core
{\em  and}
$32$-bit \CORE{1} core;
the variant  which assumes $w = 64$
(\ISE{4})
was  evaluated wrt.
{\em only}
on the
$64$-bit \CORE{1} core.
As noted, for \ISE{1}, \ISE{2} and \ISE{5} the potential for a trade-off
between latency and area exists.  Each such case is catered for through
the consideration of two optimisation goals:
the (A)rea    goal
instantiates $1$ S-box   and has a $n$-cycle execution latency,
whereas
the (L)atency goal
instantiates $4$ S-boxes and has a $1$-cycle execution latency.

\REFTAB{tab:eval:hw} 
records
various metrics 
associated with the hardware implementations, 
arranged in two parts: 
the  left-hand part relates to each ISE in isolation,
whereas 
the right-hand part relates to each ISE integrated with a given core
(noting that the first row illustrates the baseline, with {\em no} ISE).
Throughout, both area (measured in NAND2 equivalent gates) and Longest 
Topological Path (LTP) are as reported by Yosys~\cite{yosys}.  We found 
that none of the ISEs affected the critical path of either the \CORE{2} 
or \CORE{1} core.
% TODO: address the issue re. Rocket area (i.e., the caches dominating)
Considering each as implemented on \CORE{1} core, we note the overhead
wrt. area is marginal: this stems from the fact that the baseline area
of \CORE{1} includes the data and instruction caches.

% =============================================================================
