% =============================================================================

\paragraph{RISC-V.}
\label{sec:bg:riscv}

RISC-V is a (relatively) new ISA, with academic origins~\cite{riscv:1,riscv:2}.
Unlike x86 or ARMv8-A, RISC-V is a free-to-use 
open standard, managed by RISC-V International.
The base ISA is extremely simple, consisting of only $50$ instructions,
and adopts {\em strongly} RISC-oriented design principles.
RISC-V is also highly modular, having been {\em designed to be extended}.
The general-purpose base ISA can (optionally) be
supplemented using sets of special-purpose, standard or non-standard
extensions to
support additional functionality 
(e.g., floating-point, 
       via the 
       standard F~\cite[Section 11]{RV:ISA:I:19}
                and
                D~\cite[Section 12]{RV:ISA:I:19}
       extension),
or 
satisfy specific optimisation goals
(e.g., code density, 
       via the 
       standard C~\cite[Section 16]{RV:ISA:I:19}
       extension).
RISC-V International delegates the development of
extensions to a dedicated task group.
The Cryptographic Extensions Task
Group\footnote{
  \url{https://lists.riscv.org/g/tech-crypto-ext}
} provides some specific context for this paper, through their remit to 
develop scalar and vector extensions to support cryptography.

RISC-V uses $32$ registers
denoted $\GPR[*][ i ]$, for $0 \leq i < 32$,
and uses \RVXLEN to denote the register bit-width of the base ISA.
$
\GPR[*][ 0 ]
$
is fixed to $0$, while $\GPR$s $1-31$ are general purpose.
We focus on extending the
RV32I~\cite[Section 2]{RV:ISA:I:19}
and 
RV64I~\cite[Section 5]{RV:ISA:I:19},
integer RISC-V base ISA
and hence focus on systems where 
$\RVXLEN = 32$
or
$\RVXLEN = 64$.

% =============================================================================
