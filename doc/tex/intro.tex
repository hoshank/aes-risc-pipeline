% =============================================================================

\paragraph{Implementing the Advanced Encryption Standard (AES).}

In $2001$, NIST pronounced Rijndael~\cite{DaeRij:98,DaeRij:02}, 
a design due to Daemen and Rijmen, 
as winner of a $5$-year standardisation process~\cite{NBBBDFR:01} instigated 
to identify a replacement for the incumbent
Data     Encryption Standard (DES)~\cite{FIPS:46} 
block cipher; the resulting 
Advanced Encryption Standard (AES)~\cite{FIPS:197} 
was announced in $2001$.

Compared to more general cases, cryptographic kernels such as AES present a
significant challenge wrt. implementation, because they often
a) involve computationally intensive, somewhat niche functionality,
b) are deployed in a diverse range of contexts,
   and
c) form a central target in what is a complex attack surface.
The demand for efficiency, however measured, forms an overarching example
with at least two dimensions.
First,
cryptography often represents an enabling technology vs. a feature: it will
likely be viewed as overhead, therefore, when viewed from the perspective 
of a user.  Put another way,  (even {\em perceived}) {\em in}efficiency may
be problematic in terms of fitness for purpose.  Addressing this is further 
complicated by any constraints associated with the context, e.g., a demand 
for
high-volume, 
 low-latency, 
high-throughput, 
 low-footprint, 
and/or 
 low-power.
Second,
and although efficiency is a goal in and of itself therefore, it {\em also} 
acts as an enabler for security.  This is because one cannot 
(or at least {\em should} not) 
compromise security to meet efficiency requirements, implying that delivery 
of higher efficiency offers greater margin within which to deliver suitable
countermeasures against attack.

AES represents an interesting case-study wrt. techniques which attempt to 
address the challenge of efficient, secure implementation.  For example,
per the request for candidates announcement\footnote{%
\url{https://www.govinfo.gov/content/pkg/FR-1997-09-12/pdf/97-24214.pdf}
}, the AES process was instrumental in popularising a model in which
{\em both}
``security''
(e.g., resilience against cryptanalytic attack)
{\em and}
``algorithm and implementation characteristics''
(e.g., computational efficiency, memory requirements, suitability for different platform classes)
form important quality metrics for the {\em design}, in order to facilitate
techniques for higher quality {\em implementations} of it.
In addition,
both the design {\em and} implementations of it are long-lived.
From a positive perspective, 
the central importance of AES has yielded special emphasis on related
research and development effort before, during, and, most significantly, 
after the AES process.
From a negative perspective, however,
the $20+$ year period since standardisation has forced an evolution of 
implementation technique, e.g., to match evolution of both the technology 
and attack landscape.  For example,~\cite[Section 3.6]{NBBBDFR:01} covers
implementation (e.g., side-channel) attacks: this field has become richer,
and the associated threat more potent during said period.

% -----------------------------------------------------------------------------

\paragraph{Support via Instruction Set Extensions (ISEs).}

A potentially large space of implementation techniques will often exist
for a given cryptographic kernel.  One could classify instances as being 
   kernel-agnostic
   or
   kernel-specific,
and based on the use of   
   hardware              only,
                software only,
   or
   hardware and software (i.e., a hybrid of the two).
The latter class includes instances based on the concept of an
Instruction Set Extension (ISE)~\cite{GalBer:11,BarGioMar:09,RegIen:16}.
The idea is to identify, e.g., through benchmarking, a set of additional 
instructions that allow the kernel to leverage
special-purpose, domain-specific functionality
in the resulting ISE,
vs. 
general-purpose                  functionality
in the base Instruction Set Architecture (ISA),
and thereby deliver improvement wrt. pertinent quality metrics.  
Although ISEs can be an effective option for {\em both}
high(er)-end, performance-oriented
and
 low(er)-end, constrained
platforms, 
they are particularly effective for the latter by 
improving footprint and latency
vs. a software-only option
while also
improving area      and flexibility
vs. a hardware-only option.

Abstractly, an ISE design constitutes
1) an {\em interface},
   i.e., 
   addition of instructions to some 
   base ISA
   plus
2) an {\em implementation},
   i.e., 
   support for execution of said instructions through changes or addition 
   to a 
   base micro-architecture.
As a fundamental, long-lived computer systems interface, the design of or
changes to an ISA demands careful consideration
(cf.~\cite[Section 4]{Gueron:09}); this implies that the production of a 
concrete ISE design is far from trivial.  
Such a design must deliver a demonstrable improvement wrt. the kernel in 
question, {\em as well as} considering a (non-exhaustive) list of design 
goals such as

\begin{itemize}
\item minimal invasiveness,
      e.g.,
      limit the number and scope (local vs. global) of changes,
\item minimal overhead,
      e.g.,
      avoid instructions that demand support via large or complex hardware components,
\item maximal compatibility,
      e.g.,
      align with constraints in the base ISA, such as use of existing instruction formats,
      and
\item maximal       utility,
      i.e.,
      favour general- vs. {\em too} special-purpose functionality.
\end{itemize}

\noindent
By extending x86, Intel provide archetypal examples of and evidence in
support of ISEs.
For example, various generations of
non-cryptographic
Multi-Media      eXtensions (MMX),
Streaming SIMD  Extensions (SSE),
and
Advanced Vector Extensions (AVX)
support numerical kernels via vector (or SIMD) vs. scalar computation.  
Likewise, the
    cryptographic
Advanced Encryption Standard New Instructions (AES-NI)~\cite{Gueron:09,DruGueKra:19}
ISE
supports AES: it significantly improves latency and throughput
(see, e.g.,~\cite{FazLopOli:18}),
and represents a pertinent case-study wrt. the guidelines above by
a) adding just $6$ additional (vs. $1500+$ total) instructions,
b) reducing overhead by, e.g., sharing the XMM register file,
c) facilitating compatibility via the
   \VERB{CPUID}~\cite[Chapter 20]{X86:1:18}
   feature identification mechanism,
   and
d) offering utility beyond AES specifically:
   it supports AES-specific functionality but is able to maximise utility 
   through more AES-agnostic support of {\em other} kernels
   (see, e.g., use of AES-NI within SHA3~\cite{BBGR:09} and CEASER~\cite[Section 4.1]{AnkAnk:16} candidates).

% -----------------------------------------------------------------------------

\paragraph{Remit.}

In this paper, we address the challenge of supporting AES by designing and
evaluating an associated ISE for RISC-V
(see, e.g.,~\cite{riscv:1,riscv:2}).
Note that we specifically focus on extending the scalar RISC-V instruction 
set, thereby contrasting with work related to cryptographic support in the
standard V (or vector) extension~\cite[Section 21]{RV:ISA:I:19}.

On one hand, 
RISC-V represents an excellent vehicle for such work:
extensibility is a by-design feature in the ISA, whose open nature renders
exploration of such extensions easier by virtue of the range of associated 
(often open-source) implementations.  
Increased commercial deployment of such implementations suggests that work 
on RISC-V timely, and potentially of high impact.
On the other hand, however,
RISC-V also presents some unique challenges vs. previous work.
For example,
RISC-V could in fact be viewed as {\em three} related base ISAs,
 RV32I~\cite[Section 2]{RV:ISA:I:19},
 RV64I~\cite[Section 5]{RV:ISA:I:19},
and
RV128I~\cite[Section 6]{RV:ISA:I:19},
that each support a different word size:
designing an ISE that is applicable (or scales) across these options is a
complicating factor, yet an important design goal.

% TODO: need to be clearer what constraints are for a given design

% -----------------------------------------------------------------------------

\paragraph{Organisation.}

% TODO

\REFSEC{sec:bg} describes the AES cipher, existing implementation stratagies
and prior commercial and academic ISE designs.
\REFSEC{sec:design} details $5$ different candidate ISEs and the criteria
used to select and design them.
\REFSEC{sec:eval} gives evaulation results in terms of hardware cost
and software performance.
\REFSEC{sec:psca} shows how our ISE recommendation for highly resource
constrained devices can be further extended to give power side-channel
resistance.
Finally, 
\REFSEC{sec:outro}
presents some conclusions and potential directions for future work.

\ifbool{submission}{%
Note that in order to satisfy the TCHES submission guidelines, we have 
anonymised various resources and references.  We intend to open-source 
such resources post-submission, but could provide them to reviewers, 
if required, to facilitate the review process.
}{}%

% =============================================================================
