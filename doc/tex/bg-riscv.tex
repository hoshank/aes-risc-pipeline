% =============================================================================

RISC-V is a (relatively) new ISA, with its origins at UC Berkley
\cite{riscv:1}.
The core ISA is extremely simple, consisting of only $50$ instructions.
The ISA defines $32$ general purpose registers, with register $0$ always
tied to the value $0$.
The base ISA comes in $32$, $64$ and $128$-bit wide variants.
For this work, we focus on the $32$ and $64$ bit variants, since these
are the more mature and commercially relevant ones at present.

Various optional extensions are used to support more complex and specialised
functionality (e.g. the Floating point {\tt F} extension),
or to optimise for certain goals (e.g. code density and size
with the Compressed {\tt C} extension).

Further, RISC-V is a free to use, open standard managed by the independent
RISC-V Foundation.
This is in opposition to existing architectures such as ARM, which required
significant license fees to use.
This extensibility and openness makes RISC-V an excellent target for
computer architecture research.


% =============================================================================
