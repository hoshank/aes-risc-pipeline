% =============================================================================

\paragraph{Implementing the Advanced Encryption Standard (AES).}

%
% Commented out since CHES folks know the histories.
%
%In October $2000$, NIST pronounced Rijndael~\cite{DaeRij:98,DaeRij:02}, 
%a design due to Daemen and Rijmen, 
%as winner of a $5$-year standardisation process~\cite{NBBBDFR:01} instigated 
%to identify a replacement for the incumbent
%Data     Encryption Standard (DES)~\cite{FIPS:46} 
%block cipher; the resulting 
%Advanced Encryption Standard (AES)~\cite{FIPS:197} 
%was announced in $2001$.

Compared to more general workloads, cryptographic algorithms like AES
present a significant implementation challenge.
They involve computationally intensive and specialist functionality,
are used in a wide range of contexts
and
form a central target in a complex attack surface.
The demand for efficiency, is a clear example in two ways.
First,
cryptography often represents an enabling technology vs. a feature and
is often viewed as an overhead from a users perspective.
Addressing this is
complicated by constraints associated with the context, e.g., a demand 
for
high-volume, 
 low-latency, 
high-throughput, 
 low-footprint, 
and/or 
 low-power.
Second,
although efficiency is a goal in itself, it {\em also} 
acts as an enabler for security.
This is because one {\em should not}
compromise security to meet efficiency requirements.
Hence a more efficient implementation leaves greater margin to deliver
countermeasures against attack.

AES is an interesting case-study wrt. secure, efficient implementation.
For example,
per the request for candidates announcement\footnote{%
\url{https://www.govinfo.gov/content/pkg/FR-1997-09-12/pdf/97-24214.pdf}
}, the AES process was instrumental in popularising a model in which
{\em both}
``security''
(e.g., resilience against cryptanalytic attack)
{\em and}
``algorithm and implementation characteristics''
form important quality metrics for the {\em design}, in order to facilitate
techniques for higher quality {\em implementations} of it.
Additionally,
the design {\em and} implementations of AES are long-lived.
The importance of AES has led to special emphasis on related
research and development effort before, during, and, most significantly, 
after the AES process.
The $20+$ years since standardisation has forced an evolution of 
implementation techniques, to match changes in the technology 
and attack landscape.  For example,~\cite[Section 3.6]{NBBBDFR:01} covers
implementation (e.g., side-channel) attacks: this field has become richer,
and the associated threat more dangerous during said period.

% -----------------------------------------------------------------------------

\paragraph{Support via Instruction Set Extensions (ISEs).}

A large space of implementation styles often exist
for a given cryptographic algorithm.
Techniques can be
   algorithm-agnostic
   or
   algorithm-specific,
and based on the use of   
   hardware              only,
                software only,
   or
   a hybrid approach using ISEs ~\cite{GalBer:11,BarGioMar:09,RegIen:16}.
For the ISE case, the aim is to identify through benchmarking, pieces of
algorithm-specific functionality which are inefficiently represented in the
base ISA.
Said functions are then implemented in hardware, and exposed to the
programmer via one or more new instructions.

ISEs are an effective option for {\em both}
high-end, performance-oriented
and
 low-end, constrained
platforms. 
They are particularly effective for the latter where resource constraints
are tightest.
An ISE can be smaller and faster than a pure software implementation,
and more efficient in terms of performance gain per additional logic gate
than a hardware-only option.

Abstractly an ISE design constitutes
an {\em interface} to domain specific functionality through the
addition of instructions to a base ISA.
As a fundamental and long-lived computer systems interface, the design
and extension of an ISA demands careful consideration
(cf.~\cite[Section 4]{Gueron:09}). 
and the production of a concrete ISE design is not trivial.
It must deliver a quantified improvement to the workload in 
question (however measured) {\em and}
consider numerous design goals including but not limited too:

\begin{itemize}
\item Limiting the number and complexity of changes and interractions with the
    parent ISA.
\item Avoiding addition of too many instructions, or requiring large
    additional hardware modules to implement. This will hurt commercial
    adoption of the ISA.
\item Adhering to the design constraints and philosophies of the base ISA.
\item Maximising the utility of the additional functionality
      i.e.,
      favour general-purpose over special-purpose functionality.
      Special-purpose functions can be justified in terms of how frequent
      the workload is required.
      For example, though an AES ISE might {\em only}
      be useful for AES, a webserver might execute AES millions of times
      per day.
\end{itemize}

\noindent
The x86 architecture provides many examples of ISE design,
having been extended numerous times by Intel and AMD.
Various generations of
non-cryptographic
Multi-Media      eXtensions (MMX),
Streaming SIMD  Extensions (SSE),
and
Advanced Vector Extensions (AVX)
support numerical algorithms via vector (or SIMD) vs. scalar computation.  
Likewise, the
    cryptographic
Advanced Encryption Standard New Instructions (AES-NI)~\cite{Gueron:09,DruGueKra:19}
ISE
supports AES: it significantly improves latency and throughput
(see, e.g.,~\cite{FazLopOli:18}),
and represents a useful case-study in the design goals above:
It adds just $6$ additional (vs. $1500+$ total) instructions,
reduces overhead by sharing the pre-existing XMM register file,
and facilitates compatibility via the
\VERB{CPUID}~\cite[Chapter 20]{X86:1:18}
feature identification mechanism.
It is also (sometimes un-expectedly) useful beyond AES.
E.g., the Gr{\o}stl hash function ~\cite{GKMMRST:11} uses the S-box,
and
the YAES~\cite{BosVer:14} authenticated encryption scheme uses a full round.
It can even be used to accelerate the Chinese SM4 block cipher\footnote{\url{https://github.com/mjosaarinen/sm4ni}}

% =============================================================================
