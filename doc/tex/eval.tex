% =============================================================================

\begin{table}[h]
\centering
\begin{tabular}{lrrrr}
Variant     & ISE Size & ISE LTP & \CORE{2} Size & Size Overhead \\ \hline
Baseline    & -        & -       & 37362         & -             \\
V1  Size    & 2174     & 22      & 40148         & $ 7\%$        \\
V1  Speed   & 3471     & 19      & 41711         & $12\%$        \\
V2  Size    & 1306     & 21      & 38800         & $ 4\%$        \\
V2  Speed   & 3546     & 19      & 41181         & $10\%$        \\
V3          & 1157     & 30      & 38598         & $ 3\%$        \\
V5  Size    & 1772     & 23      & 39238         & $ 5\%$        \\
V5  Speed   & 4122     & 22      & 42058         & $13\%$        \\
\end{tabular}
\caption{
Hardware implementation costs based on the 32-bit \CORE{2} CPU core.
Gate counts and topological path lengths are obtained using the
Yosys\cite{yosys} tool suite.
}
\label{tab:eval:hw}
\end{table}

Table \ref{tab:eval:hw} shows the hardware implementation costs of the
different variants.
The ISE Size column records the size in NAND2 equivilent gates of each
variant, instantiated independently from any wider system.
The LTP column gives the Longest Topological Path of the synthesised
functional unit circuit from input to combinatorial output.
The \CORE{2} Size column gives the size in NAND2 equivilent gates of the
\CORE{2}, with the various AES functional units integrated.
The ``Baseline" row gives the size of the core without any of the
ISEs integrated.
We found that none of the proposed ISEs affected the critical
path of the \CORE{2} core.

% ------------------------------------------------------------

\begin{table}[h]
\begin{tabular}{l|cc|cc|cc|cc|cc}
& \multicolumn{2}{c}{Static Size} 
& \multicolumn{2}{c}{\begin{tabular}[c]{@{}c@{}}KeySchedule\\ Encrypt\end{tabular}} 
& \multicolumn{2}{c}{\begin{tabular}[c]{@{}c@{}}KeySchedule\\ Decrypt\end{tabular}}
& \multicolumn{2}{c}{\begin{tabular}[c]{@{}c@{}}AES 128\\ Encrypt\end{tabular}}
& \multicolumn{2}{c}{\begin{tabular}[c]{@{}c@{}}AES 128\\ Decrypt\end{tabular}} \\
Variant  & .text & .data & iret & cycles & iret & cycles & iret & cycles & iret & cycles \\ \hline
 TTable  & 3337  & 10250 & 483  & 589    & 1764 & 2236   & 1015 & 1142   & 1017 & 1117   \\
 Byte    &       &       &      &        &      &        &      &        &      &        \\
V1 Size  & 1793  & 10    & 251  & 420    & 257  & 438    & 595  & 906    & 597  & 857    \\
V1 Speed & 1793  & 10    & 251  & 380    & 257  & 398    & 595  & 706    & 597  & 697    \\
V2 Size  & 1536  & 10    & 251  & 397    & 388  & 817    & 298  & 645    & 301  & 647    \\
V2 Speed & 1536  & 10    & 251  & 367    & 388  & 679    & 298  & 394    & 301  & 396    \\
V3       & 2382  & 10    & 271  & 387    & 721  & 1143   & 323  & 412    & 325  & 407    \\
V5 Size  & 1769  & 10    & 385  & 535    & 391  & 553    & 310  & 674    & 312  & 652    \\
V5 Speed & 1769  & 10    & 385  & 505    & 391  & 523    & 310  & 414    & 312  & 408
\end{tabular}
\caption{
Software size and performance for reference and accelerated AES
implementations. All measurements are for AES 128. Encrypt/Decrypt columns
are for a single block.
}
\label{tab:eval:sw}
\end{table}

Table \ref{tab:eval:sw} shows performance and code size results for
each ISE, with software only versions of AES used as a baseline.

% ------------------------------------------------------------

% =============================================================================
