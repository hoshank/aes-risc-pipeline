% =============================================================================

\ISE{3}
is based on~\cite{NadIkeKur:04,BBFR:06,Saarinen:20}; it
assumes 
$\RVXLEN = 32$
and adopts a 
column-packed 
representation of state and round key matrices.

As detailed in
\REFFIG{fig:v3:mnemonics}
and
\REFFIG{fig:v3:pseudo},
\ISE{3}
adds
$ 4$
instructions ($2$ for encryption, $2$ for decryption).
The basic idea is to support an implementation strategy aligned with use
of 
T-tables~\cite[Section 4.2]{DaeRij:02}, 
but compute entries in hardware vs. storing the look-up entries in memory.
For example
\VERB{saes.v3.encsm}
extracts                     an     element from a packed column,
 applies \AESFUNC{SubBytes}  to the element,
 expands                        the element into a packed column,
 applies \AESFUNC{MixColumn},
then
 applies \AESFUNC{AddRoundKey}.
The inclusion of \AESFUNC{AddRoundKey} follows~\cite{Saarinen:20}, which
improves on~\cite{NadIkeKur:04,BBFR:06}; as a result of this,
the instruction format for
\VERB{saes.v3.encsm}
includes $2$ source and $1$ destination register address.
The requirement for $1$ application of the S-box allows for a more efficient 
functional unit than \ISE{1} or \ISE{2}, for example, either wrt. latency or 
area.

\REFFIG{fig:v3:round}
demonstrates that use of \ISE{3} to implement AES encryption requires
$20$ instructions per round:
$ 4$ \VERB{ lw}           
     instructions to load the round key,
and
$16$ \VERB{saes.v3.encsm} 
     instructions to apply \AESFUNC{SubBytes}, \AESFUNC{ShiftRows}, \AESFUNC{MixColumns}, and \AESFUNC{AddRoundKey}.
In the $Nr$-th round, which omits \AESFUNC{MixColumns},
     \VERB{saes.v3.encsm}
is replaced by 
     \VERB{saes.v3.encs}.

% =============================================================================
