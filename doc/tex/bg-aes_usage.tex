% =============================================================================

The
Electronic CodeBook (ECB) mode
~\cite[Section 6.1]{NIST:sp.800.38a} 
captures ``raw'' use of AES as is.  However, demand for enhanced security
properties and/or functionality, for example, has led to use of AES in a
variety of {\em other} constructions.
Although not our focus per se, such use-cases are subtly important in the 
sense they inform or constrain the ISE interface: if the interface cannot 
support standard constructions, for example, then it cannot be deemed fit 
for purpose even {\em if} it supports AES as is.

% -----------------------------------------------------------------------------

\subsubsection{Modes of operation: ``AES plus insignificant extras''}

Some modes of operation use AES in combination with relatively
insignificant 
additional operations
(e.g., XOR).
Pertinent examples include the following.

\begin{itemize}
\item Encryption
      and 
      decryption
      using
      Cipher Block Chaining (CBC) mode
      ~\cite[Section 6.2]{NIST:sp.800.38a}
      is outlined in
      \REFALGO{alg:cbc:enc}
      and
      \REFALGO{alg:cbc:dec}
      respectively:
      note that it uses {\em both} AES encryption {\em and} decryption.
\item Encryption
      and 
      decryption
      using
      Counter               (CTR) mode
      ~\cite[Section 6.5]{NIST:sp.800.38a}
      is outlined in
      \REFALGO{alg:ctr:enc}
      and
      \REFALGO{alg:ctr:dec}
      respectively:
      note that it uses {\em only} AES encryption, and that suitable choices of 
      the $iv$ and increment function~\cite[Appendix B]{NIST:sp.800.38a} $f$ are
      required.
\end{itemize}

% -----------------------------------------------------------------------------

\subsubsection{Modes of operation: ``AES plus   significant extras''}

Some modes of operation use AES in combination with relatively
  significant 
additional operations.  
We pay particular attention to examples which combine the use of AES with
operations in the finite field $\F_{2^{128}}$ constructed as
$
\F_{2}[\IND{x}] / ( \IND{x}^{128} + \IND{x}^{7} + \IND{x}^{2} + \IND{x} + 1 ) .
$
Field 
      addition
and
multiplication
are denoted by
$\oplus$
and
$\otimes$
respectively:
a natural encoding/decoding is assumed, allowing conversion between field 
elements and, e.g., $128$-bit plaintexts or ciphertexts.
Whereas $\oplus$ can be realised in most ISAs using XOR, $\otimes$ can be
more problematic.  This has led many ISAs to include associated ISEs for
what is often termed ``carry-less'' multiplication; 
examples include
Intel via \VERB{PCLMULQDQ}~\cite[Page 4-241]{X86:2:18},
ARM   via \VERB{PMULL} and \VERB{PMULL2}~\cite[Section C7.2.215]{ARMv8-A:20},
and
SPARC via \VERB{XMULX} and \VERB{XMULXHI}~\cite[Section 7.143]{SPARC:16}.

\begin{itemize}
\item The Galois/Counter Mode (GCM)
      ~\cite{NIST:sp.800.38d}
      supports authenticated encryption, using a parameterisable block cipher,
      e.g., as AES-GCM when using AES.
      AES-GCM is comprised of two components:
      1)    \ALG{GCTR}
            is responsible for 
                encryption,
            using AES; 
            a simplified description of
            ~\cite[Algorithm 3]{NIST:sp.800.38d}
            is presented in
            \REFALGO{alg:gctr}
            for completeness.
      2)    \ALG{GHASH}
            is responsible for
            authentication,
            using operations in $\F_{2^{128}}$;
            a simplified description of
            ~\cite[Algorithm 2]{NIST:sp.800.38d}
            is presented in
            \REFALGO{alg:ghash}
            for completeness,
            where the so-called hash key is
            $
            h = \SCOPE{\ID{AES}}{\ALG{Enc}}( k, 0 ) .
            $
\item The XTS-AES mode 
      ~\cite{NIST:sp.800.38e}
      is an AES-specific instantiation of the XEX (XOR Encrypt XOR) tweakable
      block cipher of Rogaway~\cite{Rogaway:04};
      rather than ``data in transit'', XTS-AES is designed to support access
      to ``data at rest'' on some block-addressable storage medium.
      \REFALGO{alg:xts:enc}
      and
      \REFALGO{alg:xts:dec}
      capture XTS-AES encryption and decryption respectively, noting that 
      $
      \alpha = \IND{x} .
      $
\end{itemize}

% -----------------------------------------------------------------------------

\subsubsection{Other}

Some constructions use components {\em in} AES, rather than AES {\em itself}.
For example,
the Gr{\o}stl~\cite{GKMMRST:11} hash function
(a SHA-3  candidate; see also~\cite{BBGR:09})
uses the S-box~\cite[Section 4.3]{GKMMRST:11},
and
the YAES~\cite{BosVer:14} authenticated encryption scheme
(a CAESAR candidate; see also~\cite[Section 4.1]{AnkAnk:16})
uses a full round~\cite[Section 1.3]{BosVer:14}.

% =============================================================================
