% =============================================================================

\REFSEC{sec:bg:aes_impl_ise}
outlined a range of ISE designs, demonstrating a large design space of
options that we {\em could} consider.  To narrow the design space into
those we {\em do} consider, we use the requirements outlined below:

\begin{requirement}\label{req:1}
The ISE must support
1) AES encryption {\em and} decryption,
   and
2) {\em all} parameter sets, i.e., AES-128, AES-192, and AES-256.
Support for 
auxiliary operations, e.g., key schedule, 
are an advantage but not a requirement.
\end{requirement}

\begin{requirement}\label{req:2}
The ISE must align with the wider RISC-V design principles.
This means it should 
favour simple building-block operations,
and
use instruction encodings with at most
$2$ source registers and
$1$ destination register.
This avoids the cost of a general-purpose register file with more than $2$
read ports or $1$ write port.
\end{requirement}

\begin{requirement}\label{req:3}
The ISE must use
the RISC-V general-purpose scalar register file 
to store operands and results, rather than
any vector register file.
This requirement excludes the majority of standard ISEs outlined in 
\REFSEC{sec:bg:aes_impl_ise}.
\end{requirement}

\begin{requirement}\label{req:4}
The ISE must not introduce
special-purpose       architectural state, 
nor rely on
special-purpose micro-architectural state
(e.g., caches or scratch-pad memory).
\end{requirement}

\begin{requirement}\label{req:5}
The ISE must enable data-oblivious execution of AES, preventing
timing attacks based on execution latency
(e.g., stemming from accesses to a pre-computed S-box).
\end{requirement}

\begin{requirement}
The ISE must be efficient, in terms of improvement in execution latency 
per area required: this balances the value in {\em both} metrics, vs. an 
exclusive preference for one or the other.
Efficiency wrt. 
auxiliary metrics, e.g., memory footprint or instruction encoding points,
are deemed an advantage but not a requirement.
\end{requirement}

\noindent
Overall, the requirements combine to intentionally target the ISE at 
 low(er)-end,
resource-constrained (e.g., embedded) platforms.  
We view such a focus as reasonable, because existing work on adding
cryptographic support to the
standard 
Vector extension ~\cite[Section 21]{RV:ISA:I:19}
already caters for
high(er)-end
alternatives.

We arrive at five ISE variants using the requirements, the description of 
which is split into
an 
intuitive 
description in one of the following \SEC[s]
and
a
technical
description
(e.g., a list of instructions and their semantics)
in an associated \APPX.

% =============================================================================
