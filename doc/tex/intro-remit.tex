
\paragraph{Remit and organisation.}

In this paper we address the challenge of supporting AES on the RISC-V base
ISA
(see, e.g.,~\cite{riscv:1,riscv:2}),
and therefore inform on-going efforts to standardise cryptographic ISEs for 
RISC-V.  In specific terms, our contributions are as follows:

\begin{enumerate}

\item In 
      \REFSEC{sec:bg}
      we capture some background, including a limited form of
      Systematisation of Knowledge (SoK)
      wrt. ISEs for AES.

\item In 
      \REFSEC{sec:ise}
      we implement and evaluate five different ISEs for AES on two different 
      RISC-V compliant base micro-architectures.
      As well as exploring existing ISE designs, 
      \REFSEC{sec:ise:design:v5}
      introduces what is, to the best of our knowledge, a novel ISE design 
      that leverages a quadrant-packed representation of the state.

\item In
      \REFSEC{sec:gcm}
      we evaluate how the
      standard 
      B 
      extension~\cite[Section 21]{RV:ISA:I:19}
      to RISC-V can be harnessed for efficient implementation of AES-GCM.

\item In
      \REFSEC{sec:sca}
      we select one candidate ISE design from 
      \REFSEC{sec:ise},
      and demonstrate how the associated implementation can be hardened
      against DPA-style attacks.

\end{enumerate}

\noindent
On one hand, 
RISC-V represents an excellent vehicle for such work:
extensibility is a by-design feature in the ISA, whose open nature renders
exploration of such extensions easier by virtue of the range of associated 
(often open-source) implementations.  
Increased commercial deployment of such implementations suggests that work 
on RISC-V is timely and potentially of high impact.
RISC-V also presents unique challenges vs. previous work.
For example,
RISC-V could in fact be viewed as {\em three} related base ISAs,
 RV32I~\cite[Section 2]{RV:ISA:I:19},
 RV64I~\cite[Section 5]{RV:ISA:I:19},
and
RV128I~\cite[Section 6]{RV:ISA:I:19},
that each support a different word size:
designing ISEs that are applicable (or scale) across these options is a
complicating factor.

\ifbool{submission}{%
Note that in order to satisfy the TCHES submission guidelines, we have 
anonymised various resources and references.  We intend to open-source 
such resources post-submission, but could provide them to reviewers, 
if required, to facilitate the review process.
}{}%

