% =============================================================================

By reproducing~\cite[Section 4.3]{TilGro:06},
\ISE{2}
assumes 
$w = 32$
and adopts a 
column-packed 
representation of state and round key matrices.

As detailed in
\REFFIG{fig:v2:mnemonics}
and
\REFFIG{fig:v2:pseudo},
\ISE{2}
adds
$ 4$
instructions ($2$ for encryption, $2$ for decryption).
For example
\VERB{saes.v2.encs}
\VERB{saes.v2.encs}
applies 
\AESFUNC{SubBytes}  to elements in   a packed column,
and
\VERB{saes.v2.encm}
applies 
\AESFUNC{MixColumn} to               a packed column;
the instruction format for
\VERB{saes.v2.encs}
and
\VERB{saes.v2.encm}
uses 
includes $2$ source and $1$ destination register address.
\ISE{2} improves \ISE{1} by applying \AESFUNC{ShiftRows} 
{\em implicitly}:
this is possible by careful indexing of elements in source and destination
columns during application of \AESFUNC{SubBytes} and \AESFUNC{MixColumns},
and also permits
\VERB{saes.v2.encs}
to be used within the key schedule.
The same trade-off is possible as in \ISE{1}, whereby
$n$ physical S-box instances are (re)used in $4/n$ cycles
(e.g., $1$ instance in $4$ cycles, or $4$ instances in $1$ cycle).

\REFFIG{fig:v2:round}
demonstrates that use of \ISE{2} to implement AES encryption requires
$16$ instructions per round:
$ 4$ \VERB{lw}           
     instructions to load the round key,
$ 4$ \VERB{xor}           
     instructions to apply \AESFUNC{AddRoundKey},
$ 4$ \VERB{saes.v1.encs}  
     instructions to apply \AESFUNC{SubBytes},
     and
$ 4$ \VERB{saes.v1.encm}  
     instructions to apply \AESFUNC{MixColumns}.
In the $Nr$-th round, which omits \AESFUNC{MixColumns},
\AESFUNC{ShiftRows} must be applied
{\em explicitly}
using additional 
$12$ instructions.

% =============================================================================
