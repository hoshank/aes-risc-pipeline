% =============================================================================

\subsection{Hardware Evaluation}
\label{sec:eval:sw}

Each ISE variant was evaluated on the host cores
described in \REFSEC{sec:design}.
The 32-bit designs (V1,V2,V3,V5) were implemented on both the
\CORE{1} and \CORE{2} cores.
The 64-bit design (v4) was only evaluated on the 64-bit configuration
of the \CORE{1} core.
Table \ref{tab:eval:hw} shows the hardware implementation costs.

For variants 1, 2 and 5, two implementations are evaluated.
The {\em Size} optimised implementations instantiate only a single
Forward/Inverse SBox and MixColumns circuit and take multiple cycles
to produce a result.
The {\em Latency} optimised implementations instantiate $4$ SBox and
MixColumn circuits to produce their results in a single processor 
clock cycle.

The {\em Size} columns of Table \ref{tab:eval:hw} 
record the number of NAND2 equivalent gates of each variant,
instantiated independently from any wider system.
The LTP column gives the Longest Topological Path of the synthesised
functional unit circuit from input to combinatorial output.
The \CORE{2} Size column gives the size in NAND2 equivalent gates of the
\CORE{2}, with the various AES functional units integrated.
The ``Baseline" row gives the size of the core without any of the
ISEs integrated.
We found that none of the proposed ISEs affected the critical
path of the \CORE{2} core.


% =============================================================================

\subsection{Software Evaluation}
\label{sec:eval:sw}

To evaluate the software performance, we implemented AES 128, and
measure the static code size, instruction execution counts and cycle
counts of the Key Schedule, Encrypt and Decrypt functions.
We also separate generation of the KeySchedule for Encrypt and Decrypt.
A T-table implementation of AES (See \REFSEC{sec:bg:aes_impl_sw})
is used as a performance baseline.

\REFTAB{tab:eval:sw:size} shows the static code size for each
function.
These are measured targeting the {\tt rv32imc} base ISA for all variants,
except for V4, which targets {\tt rv64imc}.
Decryption KeySchedule entries with a $\star$ indicate that the
implementation uses the Equivalent Inverse Cipher construction
\cite[Section 5.3.5, Page 23]{FIPS:197}, meaning that the
Decrypt KeySchedule calls the Encrypt KeySchedule, then performs some
additional post processing. The code size listed is for the post processing.
The V5 Encrypt Key Schedule is split into the basic KeySchedule
operation, plus a re-arrangement of the bytes in each word to make
KeyAddition more efficient in subsequent encrypt/decrypt operations.

For the Encrypt and Decrypt functions, all versions are very similar
in terms of code size.
The 32-bit variants (V1,V2,V3,V5) all use loop-based implementations.
V1, V2, V5 implement one round per loop iteration. V3 implements two
rounds per loop iteration to avoid needless register move operations.
The 64-bit variant (V4) uses a completely un-rolled implementation.

\REFTAB{tab:eval:sw:perf:scarv}
and
\REFTAB{tab:eval:sw:perf:rocket}
give cycle and instruction counts for each
variant under the \CORE{2} and \CORE{1} cores respectively.
Each implementation uses word-aligned state, meaning the input blocks
can be loaded with $4$ load-word instructions on $32$-bit host cores,
or $2$ load-double instructions on the $64$-bit host cores.
For the Encrypt/Decrypt columns, we show instruction execution counts
and processor clock cycle measurements for a single block
encryption of AES-128.
Likewise, the KeySchedule columns are for a single AES-128 KeySchedule
operation.

% ------------------------------------------------------------

%
% Hardware evaluation results.
%

\begin{table}
\centering
\begin{tabular}{lrrrr}
Variant     &     ISE Size &     ISE LTP & \CORE{2} Size   & \CORE{1} Size \\
\hline
Baseline    &     -        &     -       & 37375 ($1.00$x) &               \\
V1 (Latency)&     3472     &{\bf 19}     & 41723 ($1.12$x) &               \\
V2 (Latency)&     3547     &{\bf 19}     & 41199 ($1.10$x) &               \\
V5 (Latency)&     4121     &     22      & 42070 ($1.13$x) &               \\
V3          &{\bf 1157}    &     30      & 38610 ($1.03$x) &               \\
V4          &              &             & N/A             & {\bf TODO}    \\
V1 (Size)   &     2174     &     22      & 40161 ($1.07$x) &               \\
V2 (Size)   &     1381     &     21      & 38885 ($1.04$x) &               \\
V5 (Size)   &     1927     &     23      & 39251 ($1.05$x) &               \\
\end{tabular}
\caption{
Hardware implementation costs of each ISE.
The ISE columns list the size and depth of the ISE Verilog hardware
descriptions when instantiated in isolation.
The \CORE{2} and \CORE{1} columns give the size (in NAND2 equivalent gates)
of the ISEs when integrated into each core.
Gate counts and topological path lengths are obtained using the
Yosys\cite{yosys} tool suite.
}
\label{tab:eval:hw}
\end{table}

%
% Static Code Size Results
%

\begin{table}
\centering
\begin{tabular}{l|r|r|r|r|r}
Variant &
{\begin{tabular}[c]{@{}c@{}}AES 128 Block\\ Encrypt\end{tabular}} &
{\begin{tabular}[c]{@{}c@{}}AES 128 Block\\ Decrypt\end{tabular}} &
{\begin{tabular}[c]{@{}c@{}}KeySchedule\\ Encrypt\end{tabular}}   &
{\begin{tabular}[c]{@{}c@{}}KeySchedule\\ Decrypt\end{tabular}}   &
.data   \\ \hline
%Byte   &           &          &     312     &  -         & 522   \\
T-table &     804   &     804  &     154     & $174\star$ & 5120  \\
V1      &     424   &     424  &{\bf 68     }&  -         & 10    \\
V2      &{\bf 234 } &{\bf 238 }&{\bf 68     }& $62\star$  & 10    \\
V3      &     290   &     290  &     86      & $64\star$  & 10    \\
V4      &     268   &     268  &     168     & $100\star$ &  0    \\
V5      &     266   &     278  &    $82+208 $&  -         & 10    \\
\end{tabular}
\caption{
Static code size metrics for each variant, measured in bytes.
}
\label{tab:eval:sw:size}
\end{table}

%
% SCARV CPU Performance Results
%

\begin{table}
\centering
\begin{tabular}{l|rr|rr|rr|rr}
& \multicolumn{2}{c|}{\begin{tabular}[c]{@{}c@{}}AES 128 Block\\ Encrypt\end{tabular}}
& \multicolumn{2}{c|}{\begin{tabular}[c]{@{}c@{}}AES 128 Block\\ Decrypt\end{tabular}}
& \multicolumn{2}{c|}{\begin{tabular}[c]{@{}c@{}}KeySchedule\\ Encrypt\end{tabular}} 
& \multicolumn{2}{c }{\begin{tabular}[c]{@{}c@{}}KeySchedule\\ Decrypt\end{tabular}}
\\
Variant     &     iret &     cycles &     iret &     cycles &     iret &     cycles &     iret &     cycles \\ \hline
 T-table    &     953  &     1078   &     953  &     1058   &     445  &     555    &     1726 &     2202    \\
V1 (Latency)&     533  &     635    &     533  &     647    &{\bf 213} &{\bf 333}   &{\bf 219} &{\bf 350}    \\
V2 (Latency)&{\bf 236} &{\bf 343}   &{\bf 237} &{\bf 343}   &{\bf 213} &     344    &     350  &     656     \\
V5 (Latency)&     248  &     346    &     248  &     349    &     347  &     489    &     353  &     506     \\
V3          &     261  &     351    &     261  &     346    &     233  &     352    &     683  &     1109    \\
                                                                                                    
V1 (Size)   &     533  &     795    &     533  &     815    &{\bf 213} &     373    &     219  &     390     \\
V2 (Size)   &{\bf 236} &     580    &     237  &     580    &{\bf 213} &     374    &     350  &     794     \\
V5 (Size)   &     248  &     598    &     248  &     590    &     347  &     519    &     353  &     536  
\end{tabular}                
\caption{                    
Performance results for the \CORE{2} core.
Note the $64$-bit \ISE{4} is absent.
}
\label{tab:eval:sw:perf:scarv}
\end{table}

%
% Rocket Core Performance Results
%

\begin{table}
\centering
\begin{tabular}{l|rr|rr|rr|rr}
& \multicolumn{2}{c|}{\begin{tabular}[c]{@{}c@{}}AES 128 Block\\ Encrypt\end{tabular}}
& \multicolumn{2}{c|}{\begin{tabular}[c]{@{}c@{}}AES 128 Block\\ Decrypt\end{tabular}}
& \multicolumn{2}{c|}{\begin{tabular}[c]{@{}c@{}}KeySchedule\\ Encrypt\end{tabular}} 
& \multicolumn{2}{c }{\begin{tabular}[c]{@{}c@{}}KeySchedule\\ Decrypt\end{tabular}}
\\
Variant     &      iret &     cycles &     iret &      cycles &    iret &     cycles &     iret &     cycles\\
\hline
%Byte       &           &            &          &             &         &            &          &          \\
T-table     &     948   &      1143  &     949  &       1025  &     444  &     478    &     1726 &     1977 \\
V1 (Latency)&     528   &      685   &     529  &       680   &{\bf 212} &     341    &{\bf 214} &{\bf 290} \\
V2 (Latency)&{\bf 231}  &{\bf  359}  &{\bf 233} &       368   &{\bf 212} &{\bf 315}   &     350  &     508  \\
V5 (Latency)&     243   &      414   &     244  &{\bf   319}  &     346  &     427    &     348  &     424  \\
V3          &     253   &      445   &     254  &       445   &     233  &     470    &     674  &     2425 \\
V4          &     TODO  &            &          &             &          &            &          &          \\
V1 (Size)   &     528   &      804   &     529  &       744   &     212  &     357    &     214  &     335  \\
V2 (Size)   &{\bf 231}  &      511   &{\bf 233} &       520   &     212  &     345    &     350  &     646  \\
V5 (Size)   &     243   &      585   &     244  &       543   &     346  &     504    &     348  &     454  \\
\end{tabular}
\caption{
Performance results for the \CORE{1} core.
}
\label{tab:eval:sw:perf:rocket}
\end{table}


% =============================================================================

\subsection{Evaluation Discussion}
\label{sec:eval:results}

\begin{table}[]
\centering
\begin{tabular}{l|rr|rr|rr|rr}
& \multicolumn{2}{c}{\begin{tabular}[c]{@{}c@{}}AES 128 Block\\ Encrypt\end{tabular}}
& \multicolumn{2}{c}{\begin{tabular}[c]{@{}c@{}}AES 128 Block\\ Decrypt\end{tabular}}
& \multicolumn{2}{c}{\begin{tabular}[c]{@{}c@{}}KeySchedule\\ Encrypt\end{tabular}} 
& \multicolumn{2}{c}{\begin{tabular}[c]{@{}c@{}}KeySchedule\\ Decrypt\end{tabular}}
\\
Variant      &       iret   &       cycle &       iret  &       cycles &       iret &     cycles &       iret  & cycles   \\
\hline                                                                                                   
V1 (Latency) &        2.12  &        2.01 &        2.12 &        1.94  &       2.48 &     1.98   &        9.35 &  7.47    \\
V2 (Latency) &        4.69  &        3.65 &        4.67 &        3.58  &       2.43 &     1.87   &        5.73 &  3.90    \\
V5 (Latency) &        3.84  &        3.12 &        3.84 &        3.03  &       1.28 &     1.13   &        4.89 &  4.35    \\
V3           & {\bf 13.01}  & {\bf 10.94} & {\bf 13.01} & {\bf 10.89}  & {\bf 6.80} & {\bf 5.62} &  9.00       &  7.07    \\
V4           &        TODO  &             &             &              &            &            &             &          \\
V1 (Size)    &         3.39 &        2.57 &        3.39 &         2.46 &       3.96 &      2.82  & {\bf 14.94} &  {\bf 10.70}   \\
V2 (Size)    &        12.05 &        5.55 &       12.00 &         5.44 &       6.23 &      4.43  & 14.72       &  8.28    \\
V5 (Size)    &         8.22 &        3.86 &        8.22 &         3.83 &       2.74 &      2.29  & 10.46       &  8.79
\end{tabular}
\caption{Comparison of performance improvement unit-area for each
variant. 
}
\label{tab:eval:results}
\end{table}

For the $32$-bit ISEs 
(\ISE{1},\ISE{2},\ISE{3},\ISE{5})
our primary metric is
performance/area efficiency, since these ISEs must be sensible to
implement in small, embedded class cores.
To this, end we calculate an efficiency score by calculating the speedup
of each variant relative to the T-table baseline, and dividing it
by the normalised size of each variant
(see ``ISE Size'' column of \REFTAB{tab:eval:hw}).
\REFTAB{tab:eval:results} shows the performance / area efficiency results.
The \CORE{2} core results were used for these numbers, but the
same conclusions may be drawn from the \CORE{1} results.

We deliberately omit the size of the host core from our calculation,
as this dominates the total size of the system and detracts from
the comparison.

Qualitatively, we place more weight on the block encrypt/decrypt results
than the KeySchedule results, as one typically performs many more
block operations than KeySchedules.
In almost all cases, we find that \ISE{3} is clearly much more
efficient in terms of performance per unit area.
The size optimised version of \ISE{2} performs almost on par with
\ISE{3} when looking at performance/area efficiency in terms of
instructions retired.
When looking at actual execution times in terms of cycles however,
it is slower than \ISE{3} in all cases.

Based on these results, we believe that for $32$-bit RISC-V cores,
ISE \ISE{3} makes the best candidate for standardisation,
based on its efficiency, raw performance and simplicity to implement.
For the $64$-bit case, \ISE{4} is a fairly obvious design utilising
the $64$-bit data-path.

