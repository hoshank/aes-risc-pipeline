% =============================================================================

In a hardware-only implementation,
execution of 
AES         functionality
is 
performed by 
a bespoke hardware module (e.g., a memory-mapped co-processor),
whereas
execution of 
application functionality (i.e., whatever uses AES)
is (typically)
performed by 
a general-purpose processor core.
A large design space exists wrt. the former.
Gaj and Chodowiec~\cite[Section 3.3]{GajCho:00}
offer a reasonable overview, for example detailing
1) iterative,
2) combinatorial
   (or unrolled),
   and
3) inter- or intra-round
   pipelined
   architectures.
In the same way, a rich body of literature
(see, e.g.,~\cite{PMDW:04,GooBen:05,GajCho:09})
surveys concrete implementations on a variety of fabrics including both FPGA 
and ASIC.

Although not our focus per se, associated techniques are important because
they inform later content in (at least) two ways.
First,
they inform the ISE interface.
For example, some ISEs can be characterised as offering an interface to
hardware constituting one round 
(i.e., aligned with an iterative hardware implementation).
Second,
they inform the ISE implementation.
For example, a significant body of work focuses on efficient hardware 
implementation of the S-box
(see, e.g.,~\cite{Canright:05,BoyPer:12,ReyTahAsh:18}).

% =============================================================================
