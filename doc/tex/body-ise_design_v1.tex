% =============================================================================

By reproducing~\cite[Section 4.2]{TilGro:06},
\ISE{1}
assumes 
$\RVXLEN = 32$
and adopts a 
column-packed 
representation of state and round key matrices.

As detailed in
\REFFIG{fig:v1:mnemonics}
and
\REFFIG{fig:v1:pseudo},
\ISE{1}
adds
$ 4$
instructions ($2$ for encryption, $2$ for decryption).
For example,
\VERB{saes.v1.encs}
applies 
\AESFUNC{SubBytes}  
to elements in   a packed column,
and
\VERB{saes.v1.encm}
applies 
\AESFUNC{MixColumn} 
to               a packed column;
the instruction format for
\VERB{saes.v1.encs}
and
\VERB{saes.v1.encm}
includes $1$ source and $1$ destination register address.
Since 
\VERB{saes.v1.encs}
requires $4$ applications of the S-box, a trade-off between latency and
area is possible st. 
$n$ physical S-box instances are (re)used in $4/n$ cycles
(e.g., $1$ instance in $4$ cycles, or $4$ instances in $1$ cycle).

\REFFIG{fig:v1:round}
demonstrates that use of \ISE{1} to implement AES encryption requires
$47$ instructions per round:
$ 4$ \VERB{lw}           
     instructions to load the round key,
$ 4$ \VERB{xor}           
     instructions to apply \AESFUNC{AddRoundKey},
$ 4$ \VERB{saes.v1.encs}  
     instructions to apply \AESFUNC{SubBytes},
$31$ instructions to apply \AESFUNC{ShiftRows},
and
$ 4$ \VERB{saes.v1.encm}  
     instructions to apply \AESFUNC{MixColumns}.

% =============================================================================
