% =============================================================================

\paragraph{RISC-V}
\label{sec:bg:riscv}

RISC-V is a (relatively) new ISA, with academic origins~\cite{riscv:1,riscv:2}.
Unlike alternatives such as x86 or ARMv8-A, RISC-V is a free to use, 
open standard: this is managed by the independent RISC-V Foundation.
The base ISA is extremely simple, consisting of only $50$ instructions,
and adopts {\em strongly} RISC-oriented design principles.  However, it
is also highly modular: a general-purpose base ISA can (optionally) be
supplemented using set of special-purpose, standard or non-standard (or
custom) extensions to
support additional functionality 
(e.g., floating-point, 
       via the 
       standard F~\cite[Section 11]{RV:ISA:I:19}
                and
                D~\cite[Section 12]{RV:ISA:I:19}
       extension),
or 
satisfy specific optimisation goals
(e.g., code density, 
       via the 
       standard C~\cite[Section 16]{RV:ISA:I:19}
       extension).
The RISC-V Foundation delegates the development of such extensions to a
dedicated task group.  The Cryptographic Extensions Task Group\footnote{
  \url{https://lists.riscv.org/g/tech-crypto-ext}
} provides some specific context for this paper, through their remit to 
develop scalar and vector extensions to support cryptography.

We focus wlog. on extending either
RV32I~\cite[Section 2]{RV:ISA:I:19}
or
RV64I~\cite[Section 5]{RV:ISA:I:19},
i.e., the
$32$-bit 
or
$64$-bit 
integer RISC-V base ISA.
Let $\GPR[*][ i ]$, for $0 \leq i < 32$, denote the $i$-th entry of the 
General-Purpose Register (GPR) file.  Note that 
$
\GPR[*][ 0 ] = 0 ,
$
meaning the $0$-th register is fixed to $0$.
RISC-V uses \RVXLEN to denote the word size; we adopt the same approach, 
but by focusing on 
RV32I 
or
RV64I 
assume a focus on 
$\RVXLEN = 32$
or
$\RVXLEN = 64$.

% =============================================================================
