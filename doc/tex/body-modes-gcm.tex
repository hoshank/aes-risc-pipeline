% =============================================================================

\begin{adjustbox}{center,caption={Instruction    counts for multiplication in $\F_{2^{128}}$ as used by \ALG{GHASH}.
                                 },label={tab:gcm:instrs},float={table}[!t]}
\centering
\begin{tabular}{|c|c|c|rrrrrr|}
\hline
ISA    & Karatsuba & Reduce & \VERB{grev}
                            & \VERB{xor}
                            & \VERB{s[lr]li}
                            & \VERB{clmul} 
                            & \VERB{clmulh}
                            & Total \\
\hline
\hline
RV32IB &        no &    mul &$  4$&$ 36$&$  0$&$ 20$&$ 20$&$ 80$ \\
RV32IB &        no &  shift &$  4$&$ 56$&$ 24$&$ 16$&$ 16$&$116$ \\
RV32IB &       yes &    mul &$  4$&$ 52$&$  0$&$ 13$&$ 13$&$ 82$ \\
RV32IB &       yes &  shift &$  4$&$ 72$&$ 24$&$  9$&$  9$&$118$ \\
\hline
RV64IB &        no &    mul &$  2$&$ 10$&$  0$&$  6$&$  6$&$ 24$ \\
RV64IB &        no &  shift &$  2$&$ 20$&$ 12$&$  4$&$  4$&$ 42$ \\
RV64IB &       yes &    mul &$  2$&$ 14$&$  0$&$  5$&$  5$&$ 26$ \\
RV64IB &       yes &  shift &$  2$&$ 24$&$ 12$&$  3$&$  3$&$ 44$ \\
\hline
\end{tabular}
\end{adjustbox}

\begin{adjustbox}{center,caption={Modelled cycle counts for multiplication in $\F_{2^{128}}$ as used by \ALG{GHASH}.
                                 },label={tab:gcm:cycles},float={table}[!t]}
\centering
\begin{tabular}{|c|c|c|rrrr|}
\hline
ISA    & Karatsuba & Reduce & $1$-cycle       & $2$-cycle       & $3$-cycle       & $6$-cycle       \\
       &           &        & \VERB{clmul[h]} & \VERB{clmul[h]} & \VERB{clmul[h]} & \VERB{clmul[h]} \\
\hline
\hline
RV32IB &        no &    mul &     \bftab  80  &            120  &            160  &            280  \\
RV32IB &        no &  shift &            116  &            148  &            180  &            276  \\
RV32IB &       yes &    mul &             82  &    \bftab  108  &     \bftab 134  &            212  \\
RV32IB &       yes &  shift &            118  &            136  &            154  &     \bftab 208  \\
\hline
RV64IB &        no &    mul &     \bftab  24  &    \bftab   36  &             48  &             84  \\
RV64IB &        no &  shift &             42  &             50  &             58  &             82  \\
RV64IB &       yes &    mul &             26  &    \bftab   36  &     \bftab  46  &             76  \\
RV64IB &       yes &  shift &             44  &             50  &             56  &     \bftab  74  \\
\hline
\end{tabular}
\end{adjustbox}

% -----------------------------------------------------------------------------

Having dealt with efficient implementation of AES and hence \ALG{GCTR} in
\REFSEC{sec:ise}, we turn our attention to \ALG{GHASH}.  
Rather than further 
embellish the ISE for AES, we instead focus on re-use of the existing
standard 
B~\cite[Section 17]{RV:ISA:I:19}
extension.

% -----------------------------------------------------------------------------

\paragraph{Implementation.}

\ALG{GHASH}~\cite[Section 6.4]{NIST:sp.800.38d} is a universal hash defined 
over $\F_{2^{128}}$.
Conversion of the input into the correct endian'ness can be realised using
the 
\VERB{grev} (or generalised reverse)
instruction,
which reverses the bits in each byte of an input word:
$4$ (resp. $2$) 
\VERB{grev} 
instructions are therefore required on RV32IB (resp. RV64IB).
Beyond this, operations in $\F_{2^{128}}$ dominate.
Addition       in $\F_{2^{128}}$ 
is trivial, since addition modulo $2$ is equivalent to XOR: this
$4$ (resp. $2$) 
\VERB{xor} 
instructions are therefore required on RV32IB (resp. RV64IB).
Multiplication in $\F_{2^{128}}$ 
can be decomposed into two steps:
1) a $( 128 \times 128 )$-bit polynomial multiplication, 
   followed by 
2) a reduction of the $256$-bit result modulo
   $
   \IND{x}^{128} + \IND{x}^{7} + \IND{x}^{2} + \IND{x} + 1 .
   $
The first  step 
can be realised using pairs of ``carry-less'' multiplication instructions
\VERB{clmul} and \VERB{clmulh}.
These instructions compute the least-significant (resp. most-significant) 
half of a carry-less product (i.e., product over $\F_2$); pairs of 
\VERB{clmul} and \VERB{clmulh}
should be scheduled adjacently, st. capable micro-architectures are able 
to fuse them.
Use of a school-book approach 
requires
$16$ (resp. $4$) such pairs 
on RV32IB (resp. RV64IB);
optimisation using the Karatsuba method
requires
$ 9$ (resp. $3$) such pairs 
on RV32IB (resp. RV64IB),
plus some additional \VERB{xor} instructions.
The second step
can be realised using two approaches:
1) a          shift-based reduction, made possible by the low Hamming weight of the primitive polynomial,
   or
2) a multiplication-based reduction, analogous to the Montgomery or Barret methods.
Overall, the most efficient approach depends on the relative execution 
latency of
\VERB{clmul[h]}
vs.
\VERB{xor} and \VERB{s[lr]li}.
We note that 
\VERB{clmul[h]}
{\em must} exhibit data-oblivious execution latency 
(e.g., avoid data-dependent optimisations such as early-termination)
to avoid associated side-channel attacks (cf.~\cite{GOPT:09}).

% -----------------------------------------------------------------------------

\paragraph{Discussion.}

\REFTAB{tab:gcm:instrs} 
lists instruction counts for 
multiplication in $\F_{2^{128}}$,
as implemented using different combinations of the base ISA, and approaches
for the polynomial multiplication and reduction steps.
\REFTAB{tab:gcm:cycles} 
then model the execution latency 
(measured in cycles)
assuming that \VERB{grev}, \VERB{xor}, and \VERB{s[lr]li} take $1$ cycle.
Although the model only considers an in-order core in line with those used
in \REFSEC{sec:ise} and is focused on execution latency
(vs. other pertinent metrics, such as code footprint),
there are two obvious conclusions:
1) iff.
   \VERB{clmul[h]}
   has $2$ (or more) times the latency of
   \VERB{xor} and \VERB{s[lr]li},
   a 
   Karatsuba-Ofman 
   polynomial multiplication
   is preferable,
   and
2) if
   \VERB{clmul[h]}
   has $6$ (or more) times the latency of
   \VERB{xor} and \VERB{s[lr]li},
   a shift-based 
   reduction 
   is preferable.

The authors recommend the carry-less multiply instructions
specified in the proposed RISC-V B extension also be included in the
RISC-V cryptography extension.
Implementers would otherwise need to implement (a subset of) the B
extension, potentially adding functionality and cost that is not
nessesary.

% =============================================================================
