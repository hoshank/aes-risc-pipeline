% =============================================================================

Each ISE variant was integrated into the $3$ host cores 
described in \REFSEC{sec:ise:imp}.
The variants which assume  $\RVXLEN = 32$
(\ISE{1}, \ISE{2}, \ISE{3}, and \ISE{5}) 
were evaluated
on {\em both} the
$32$-bit \CORE{2} core
{\em  and} the
$32$-bit \CORE{1} core;
the variant  which assumes $\RVXLEN = 64$
(\ISE{4})
was  evaluated
on {\em only} the
$64$-bit \CORE{1} core.
For \ISE{1}, \ISE{2} and \ISE{5} a trade-off
between latency and area exists. 
Each such case is considered through two optimisation goals:
the (A)rea    goal
instantiates $1$ S-box   and has a $n$-cycle execution latency,
whereas
the (L)atency goal
instantiates $4$ S-boxes and has a $1$-cycle execution latency.

\REFTAB{tab:eval:hw} 
records
various metrics 
associated with the hardware implementations, 
arranged in two parts: 
the  left-hand part relates to each ISE in isolation,
whereas 
the right-hand part relates to each ISE integrated with a given core.
Throughout, both area (measured in NAND2 equivalent gates) and Circuit
Depth are as reported by Yosys~\cite{yosys}.  We found 
that none of the ISEs affected the critical path of either the \CORE{2} 
or \CORE{1} core.
% TODO: address the issue re. Rocket area (i.e., the caches dominating)
Considering each ISE as implemented on the \CORE{1} core, we note the 
overhead wrt. area is marginal: this stems from the fact that the 
baseline area of \CORE{1} includes the data and instruction caches.

% =============================================================================
