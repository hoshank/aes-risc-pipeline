% =============================================================================

We evaluated each ISE variant by constructing an associated implementation
of AES (recalling this means AES-128), including
$\ALG{Enc}$
and
$\ALG{Dec}$,
{\em plus}
$\ALG{Enc-KeyExp}$
and
$\ALG{Dec-KeyExp}$.
Aset of reference, {\em non}-ISE 
(e.g., T-table) 
implementations were used as a baseline.
The variants which assume  $\RVXLEN = 32$
(\ISE{1}, \ISE{2},     \ISE{3}, and \ISE{5})
uses a    rolled strategy wrt. loops:
 \ISE{1}, \ISE{2},              and \ISE{5}
use  $1$ round  per-iteration,
whereas
                       \ISE{3}
uses $2$ rounds per-iteration
to avoid needless register move operations.
The variant  which assumes $\RVXLEN = 64$
(\ISE{4})
uses an unrolled strategy.
In all cases the state is aligned\footnote{%
RISC-V does not mandate support for misaligned loads and stores, so
aligning the state this way ensures the best performance across all
cores.
} naturally, meaning any input (resp. output) can be loaded (resp. stored) 
using 
$4$ \VERB{lw} instructions on a $32$-bit core
or
$2$ \VERB{ld} instructions on a $64$-bit core.

\REFTAB{tab:eval:sw:size} 
records
memory footprint (i.e., code footprint, and static data footprint)
associated with the software implementations.
Where an entry for
\ALG{Dec-KeyExp}
is     zero, this implies that
$\ALG{Enc-KeyExp} =    \ALG{Dec-KeyExp}$
so there is no additional implementation and so no overhead.
Where an entry for
\ALG{Dec-KeyExp}
is non-zero, this implies that
$\ALG{Enc-KeyExp} \neq \ALG{Dec-KeyExp}$,
and, in particular, that the equivalent inverse cipher construction~\cite[Section 5.3.5]{FIPS:197}
is used: this allows $\ALG{Dec-KeyExp}$ to
1) call $\ALG{Enc-KeyExp}$,
   then
2) perform some additional post processing,
with the quoted footprint therefore reflecting the latter only.  
\REFTAB{tab:eval:sw:perf:2}
and
\REFTAB{tab:eval:sw:perf:1}
record
instruction (i.e., iret) cycle count
associated with the software implementations,
as executed on the \CORE{2} and \CORE{1} cores respectively.

% =============================================================================
