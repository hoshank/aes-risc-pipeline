% =============================================================================

\REFSEC{sec:bg:aes_impl_ise}
outlined a range ISE designs, which constitute a large design space of
options that we {\em could} consider.  To narrow the design space into
those we {\em do} consider, we use the requirements outlined below:

\begin{requirement}\label{req:1}
The ISE should support
1) AES encryption {\em and} decryption,
   and
2) {\em all} parameter sets, i.e., AES-128, AES-192, and AES-256.
Support for 
auxiliary operations, e.g., key schedule, 
are deemed an advantage but not a requirement per se.
\end{requirement}

\begin{requirement}\label{req:2}
The ISE should align with the wider RISC-V ethos and design principles.
Foremost, this means it should 
1) favour simple building-block operations,
   and
2) permit an instruction format with at most $2$ source and $1$ destination register address,
   and therefore avoid demanding a general-purpose register file with more than $2$ read or
   $1$ write port.
\end{requirement}

\begin{requirement}\label{req:3}
The ISE should operate on 
the RISC-V general-purpose scalar register file 
(i.e., st. $w \in \SET{ 32, 64 }$),
vs. 
any                        vector register file
(e.g., $w \ge 128$):
this requirement excludes the majority of standard ISEs outlined in 
\REFSEC{sec:bg:aes_impl_ise}.
\end{requirement}

\begin{requirement}\label{req:4}
The ISE should introduce no
special-purpose       architectural state, 
nor rely on
special-purpose micro-architectural state
(e.g., caches or scratch-pad memory).
\end{requirement}

\begin{requirement}\label{req:5}
The ISE should afford data-oblivious execution of AES, and thus prevent 
(digital) side-channel attacks based on execution time 
(e.g., stemming from accesses to the S-box).
\end{requirement}

\begin{requirement}
The ISE must be efficient, in terms of improvement in execution latency 
per area required: this balances the value in {\em both} metrics, vs. an 
exclusive preference for one or the other.
Efficiency wrt. 
auxiliary metrics, e.g., memory footprint or instruction encoding points,
are deemed an advantage but not a requirement per se.
\end{requirement}

\noindent
Overall, the requirements combine to intentionally target the ISE at 
 low(er)-end,
resource-constrained (e.g., embedded) platforms.  
We view such a focus as reasonable, because existing work on adding
cryptographic support to the
standard 
V~\cite[Section 21]{RV:ISA:I:19}
extension
already caters for
high(er)-end
alternatives.

We arrive at five ISE variants using the requirements, the description of 
which is split into
1) an 
   intuitive 
   description in one of the following \SEC[s],
   plus
2) a
   technical
   description
   (e.g., a complete list of instructions and their semantics)
   in an associated \APPX.

% =============================================================================
