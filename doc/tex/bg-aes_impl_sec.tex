% =============================================================================

Although the security of AES against a cryptanalytic attack is defined by
the design, and so out of scope, the concept of implementation attacks is
of central importance.
In essence, an implementation attack focuses on
the concrete, in practice implementation
rather than
the abstract, on paper     specification:
fundamentally, this suggests countermeasures against such attacks must be
considered alongside the implementation they relate to.
Since AES is an important target, a significant body of literature exists
wrt. implementation attacks on it: this includes both
 active (e.g., fault injection)
or
passive (i.e., side-channel),
attack techniques,
with the latter sub-divided into those dependent on
analogue,
e.g., power-based~\cite{ManOswPop:07},
or
discrete, 
e.g.,  time-based~\cite{KoeQui:99},
leakage.

From a positive perspective, use of ISEs
{\em can} provide some inherent protection against certain attacks.
For example,
ISEs typically yield constant (i.e., data-oblivious) execution latency,
so prevent some classes of time- or micro-architectural
(see, e.g.,~\cite[Section 4]{Szefer:19} and ~\cite[Section 4]{GYCH:18})
attack techniques.
From a negative perspective, however,
use of ISEs presents some unique challenges.
For example, 
Saab et al.\cite{SaaRohHam:16}
discuss power-based attacks on ISEs, AES-NI specifically; they conclude
that naive use of AES-NI will yield exploitable leakage.  Mitigation of
such leakage demands that the ISE is flexible enough to 
a) cater for instances where the leakage stems from ``inside'' the ISE,
   and, either way,
b) compose with an appropriate
   (e.g., hiding~\cite[Chapter 7]{ManOswPop:07} or masking~\cite[Chapter 10]{ManOswPop:07})
   countermeasure.
Tillich et al.~\cite{TilHerMan:07}
consider this problem to some extent, including an ISE-based option in
their investigation of hardened AES implementations, but the challenge
of developing suitable ISEs wrt. this quality metric is under-studied
in general.

% =============================================================================
