% =============================================================================

\ISE{4}
is similar to the SPARC~\cite[Page 109]{SPARC:16} ISE; it
assumes 
$\RVXLEN = 64$
and adopts a 
{\em double}
column-packed 
representation of state and round key matrices,
i.e., {\em two} columns (or $8$ elements) are packed into a $64$-bit word.
While still adhering to a format that
includes $2$ source and $1$ destination register address,
a single instruction can therefore 
1) accept  all  of the current state as  input,
   and
2) produce half of the next    state as output.

SPARC~\cite[Page 109]{SPARC:16}
adds
$ 9$
instructions ($4$ for encryption, $4$ for decryption, and $1$ auxiliary).
For example
\VERB{AES_EROUND01}
and
\VERB{AES_EROUND23}
produce
columns $0$ and $1$
and
columns $2$ and $3$
respectively.
As detailed in
\REFFIG{fig:v4:mnemonics}
and
\REFFIG{fig:v4:pseudo},
\ISE{4}
refines this slightly by 
adding 
$ 7$
instructions ($2$ for encryption, $2$ for decryption, and $3$ auxiliary).
For example
\VERB{saes.v4.encs}
applies
\AESFUNC{SubBytes}, \AESFUNC{ShiftRow}, and \AESFUNC{MixColumn}  
to elements in   a packed column,
but differs from 
\VERB{AES_EROUND01}
and
\VERB{AES_EROUND23},
because
1) it constitutes
   $1$ (vs. $2$)
   instruction,
   which is possible by observing that swapping the inputs allows 
   computation of either 
   columns $0$ and $1$ 
   or 
   columns $2$ and $3$,
   and
2) it uses 
   $2$ (vs. $3$)
   source register addresses, 
   as a result of opting not to include
   \AESFUNC{AddRoundKey}.

\REFFIG{fig:v4:round}
demonstrates that use of \ISE{4} to implement AES encryption requires
$ 6$ instructions per round:
$ 2$ \VERB{ld}           
     instructions to load the round key,
$ 2$ \VERB{xor}           
     instructions to apply \AESFUNC{AddRoundKey},
$ 2$ \VERB{saes.v4.encsm}  
     instructions to apply \AESFUNC{SubBytes}, \AESFUNC{ShiftRows}, and \AESFUNC{MixColumns}.
In the $Nr$-th round, which omits \AESFUNC{MixColumns},
     \VERB{saes.v4.encsm}
is replaced by 
     \VERB{saes.v4.encs}.

% =============================================================================
